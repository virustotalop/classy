\documentclass{article}
\usepackage{hyperref}
\usepackage{listings}
\hypersetup{
colorlinks=true
}
\title{Shitty Java \\ {\large A not so comprehensive guide to programming in Java}}
\author{Steven}
\begin{document}

\begin{titlepage}
	\maketitle
\end{titlepage}

\tableofcontents
\newpage

\section{Hello World}
\subsection{Setup}
Programming can be done with your ordinary text-editor but many prefer to use an
\href{https://wikipedia.org/wiki/Integrated_development_environment}{integrated environment or IDE} for short. A few commonly use environments are:
\begin{itemize}
	\item \href{https://www.eclipse.org/downloads/}{Eclipse}
	\item \href{https://www.jetbrains.com/idea/download/}{Intellij Idea}
\end{itemize}
Before we get ahead of ourselves we need to install the java development kit or jdk first. Go \href{https://www.oracle.com/technetwork/java/javase/downloads/index.html}{here} to get the latest release of the jdk. Once the jdk is installed you can then move onto picking your choice of IDE.

\subsubsection{Installing Eclipse}
Lorem ipsum dolor sit amet, consectetur adipiscing elit, sed do eiusmod tempor incididunt ut labore et dolore magna aliqua. Ut enim ad minim veniam, quis nostrud exercitation ullamco laboris nisi ut aliquip ex ea commodo consequat. Duis aute irure dolor in reprehenderit in voluptate velit esse cillum dolore eu fugiat nulla pariatur. Excepteur sint occaecat cupidatat non proident, sunt in culpa qui officia deserunt mollit anim id est laborum.


\subsubsection{Installing Intellij Idea}
Lorem ipsum dolor sit amet, consectetur adipiscing elit, sed do eiusmod tempor incididunt ut labore et dolore magna aliqua. Ut enim ad minim veniam, quis nostrud exercitation ullamco laboris nisi ut aliquip ex ea commodo consequat. Duis aute irure dolor in reprehenderit in voluptate velit esse cillum dolore eu fugiat nulla pariatur. Excepteur sint occaecat cupidatat non proident, sunt in culpa qui officia deserunt mollit anim id est laborum.

\subsection{Writing Hello World}
Now you are ready to explore the \emph{world of programming}. To start you will have
to make a java class file. This process is a little bit different depending on the IDE you installed earlier.

\subsubsection{Create a class in Eclipse}
Lorem ipsum dolor sit amet, consectetur adipiscing elit, sed do eiusmod tempor incididunt ut labore et dolore magna aliqua. Ut enim ad minim veniam, quis nostrud exercitation ullamco laboris nisi ut aliquip ex ea commodo consequat. Duis aute irure dolor in reprehenderit in voluptate velit esse cillum dolore eu fugiat nulla pariatur. Excepteur sint occaecat cupidatat non proident, sunt in culpa qui officia deserunt mollit anim id est laborum.



\subsubsection{Creating a class in Intellij Idea}
Lorem ipsum dolor sit amet, consectetur adipiscing elit, sed do eiusmod tempor incididunt ut labore et dolore magna aliqua. Ut enim ad minim veniam, quis nostrud exercitation ullamco laboris nisi ut aliquip ex ea commodo consequat. Duis aute irure dolor in reprehenderit in voluptate velit esse cillum dolore eu fugiat nulla pariatur. Excepteur sint occaecat cupidatat non proident, sunt in culpa qui officia deserunt mollit anim id est laborum.



\subsubsection{Hello World Explained}
\begin{lstlisting}
public static void main(String[] args)
{
   System.out.println("Hello World!");
}
\end{lstlisting}




\section{If Then Else}
If and else statements are the basics of how you make \href{https://wikipedia.org/wiki/Conditional_(computer_programming)}{conditional decisions in programming}.


\subsubsection{Conditions Expanded}




\end{document}